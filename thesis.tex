\documentclass[fontsize=12pt, paper=a4, headinclude, twoside=false, parskip=half+, pagesize=auto, numbers=noenddot, open=right, toc=listof, toc=bibliography, hidelinks]{scrreprt}
% PDF-Kompression
\pdfminorversion=5
\pdfobjcompresslevel=1
% Allgemeines
\usepackage[automark]{scrpage2} % Kopf- und Fußzeilen
\usepackage{amsmath,marvosym} % Mathesachen
\usepackage[T1]{fontenc} % Ligaturen, richtige Umlaute im PDF
\usepackage[utf8]{inputenc}% UTF8-Kodierung für Umlaute usw
\usepackage{booktabs} % Tabellen
\usepackage{tablefootnote}
% Schriften
\usepackage{mathpazo} % Palatino für Mathemodus
%\usepackage{mathpazo,tgpagella} % auch sehr schöne Schriften
\usepackage{setspace} % Zeilenabstand
\onehalfspacing % 1,5 Zeilen
% Schriften-Größen
\setkomafont{chapter}{\Huge\rmfamily} % Überschrift der Ebene
\setkomafont{section}{\Large\rmfamily}
\setkomafont{subsection}{\large\rmfamily}
\setkomafont{subsubsection}{\large\rmfamily}
\setkomafont{chapterentry}{\large\rmfamily} % Überschrift der Ebene in Inhaltsverzeichnis
\setkomafont{descriptionlabel}{\bfseries\rmfamily} % für description-Umgebungen
\setkomafont{captionlabel}{\small\bfseries}
\setkomafont{caption}{\small}
% Sprache: Deutsch
\usepackage[ngerman]{babel} % Silbentrennung
% PDF
\usepackage[ngerman,pdfauthor={Georg Donner},  pdfauthor={Georg Donner}, pdftitle={Themenspezifische Gruppierung deutscher Online-Zeitungen mit Natural Language Processing}, breaklinks=false,baseurl={}]{hyperref}
\usepackage[final]{microtype} % mikrotypographische Optimierungen
\usepackage{url} % ermögliche Links (URLs)
\usepackage{pdflscape} % einzelne Seiten drehen können
% Tabellen
\usepackage{multirow} % Tabellen-Zellen über mehrere Zeilen
\usepackage{multicol} % mehrere Spalten auf eine Seite
\usepackage{tabularx} % für Tabellen mit vorgegeben Größen
\usepackage{longtable} % Tabellen über mehrere Seiten
\usepackage{array}
%  Bibliographie
\usepackage{bibgerm} % Umlaute in BibTeX
% Bilder
\usepackage{graphicx} % Bilder
\usepackage{color} % Farben
\graphicspath{{images/}} % Lege den Standardpfad mit Bilder fest
\DeclareGraphicsExtensions{.pdf,.png,.jpg} % bevorzuge pdf-Dateien vor den anderen
\usepackage{subcaption}  % mehrere Abbildungen nebeneinander/übereinander
\usepackage[all]{hypcap} % Beim Klicken auf Links zum Bild und nicht zu Caption gehen
% Bildunterschrift
\setcapindent{0em} % kein Einrücken der Caption von Figures und Tabellen
\setcapwidth{0.9\textwidth} % Breite der Caption nur 90% der Textbreite, damit sie sich vom restlichen Text abhebt
\setlength{\abovecaptionskip}{0.2cm} % Abstand der zwischen Bild- und Bildunterschrift
% Quellcode
% für Formatierung in Quelltexten, hier im Anhang
\usepackage{listings}
\definecolor{grau}{gray}{0.25}
\lstset{
	extendedchars=true,
	basicstyle=\tiny\ttfamily,
	%basicstyle=\footnotesize\ttfamily,
	tabsize=2,
	keywordstyle=\textbf,
	commentstyle=\color{grau},
	stringstyle=\textit,
	numbers=left,
	numberstyle=\tiny,
	% für schönen Zeilenumbruch
	breakautoindent  = true,
	breakindent      = 2em,
	breaklines       = true,
	postbreak        = ,
	prebreak         = \raisebox{-.8ex}[0ex][0ex]{\Righttorque},
}
% linksbündige Fußboten
\deffootnote{1.5em}{1em}{\makebox[1.5em][l]{\thefootnotemark}}

\typearea{14} % typearea berechnet einen sinnvollen Satzspiegel (das heißt die Seitenränder usw.) siehe auch http://www.ctan.org/pkg/typearea. Diese Berechnung befindet sich am Schluss, damit die Einstellungen von oben berücksichtigt werden

\usepackage{scrhack} % Vermeidung einer Warnung


% Eigene Befehle %%%%%%%%%%%%%%%%%%%%%%%%%%%%%%%%%%%%%%%%%%%%%%%%%5
% Matrix
\newcommand{\mat}[1]{
      {\textbf{#1}}
}
\newcommand{\todo}[1]{
      {\colorbox{red}{ TODO: #1 }}
}
\newcommand{\todotext}[1]{
      {\color{red} TODO: #1} \normalfont
}
\newcommand{\info}[1]{
      {\colorbox{blue}{ (INFO: #1)}}
}
% Hinweis auf Programme in Datei
\newcommand{\datei}[1]{
      {\ttfamily{#1}}
}
\newcommand{\code}[1]{
      {\ttfamily{#1}}
}
% bild mit defnierter Breite einfügen
\newcommand{\bild}[4]{
  \begin{figure}[!hbt]
    \centering
      \vspace{1ex}
      \includegraphics[width=#2]{images/#1}
      \caption[#4]{\label{img.#1} #3}
    \vspace{1ex}
  \end{figure}
}
% bild mit eigener Breite
\newcommand{\bilda}[3]{
  \begin{figure}[!hbt]
    \centering
      \vspace{1ex}
      \includegraphics{images/#1}
      \caption[#3]{\label{img.#1} #2}
      \vspace{1ex}
  \end{figure}
}

% Bild todo
\newcommand{\bildt}[2]{
  \begin{figure}[!hbt]
    \begin{center}
      \vspace{2ex}
	      \includegraphics[width=6cm]{images/todobild}
      %\caption{\label{#1} \color{red}{ TODO: #2}}
      \caption{\label{#1} \todotext{#2}}
      %{\caption{\label{#1} {\todo{#2}}}}
      \vspace{2ex}
    \end{center}
  \end{figure}
}
 % Importiere die Einstellungen aus der Präambel
% hier beginnt der eigentliche Inhalt
\begin{document}
\pagenumbering{Roman} % Seitenummerierung mit großen römischen Zahlen 
\pagestyle{empty} % keine Kopf- oder Fußzeilen auf den ersten Seiten

% Titelseite
\clearscrheadings\clearscrplain
\begin{center}
\begin{Huge}
Themenspezifische Gruppierung deutscher Online-Zeitungen mit Natural Language Processing\\
\end{Huge}

\vspace{8mm}
Bachelorarbeit\\
\vspace{0.4cm}
\vspace{2 cm}
Georg Donner\\
Matrikel-Nummer 553821\\
\vspace{8cm}
\begin{tabular}{rl}
{\bfseries Betreuer} & Prof. Dr. Gefei Zhang\\
{\bfseries Erstprüfer} & Prof. Dr. Gefei Zhang\\
{\bfseries Zweitprüfer} & Prof. Dr. Barne Kleinen\\
\end{tabular}

\end{center}
\clearpage

\pagestyle{useheadings} % normale Kopf- und Fußzeilen für den Rest

\tableofcontents % erstelle hier das Inhaltsverzeichnis
\listoffigures % erstelle hier das Abbildungsverzeichnis
\clearpage

% richtiger Inhalt
\pagenumbering{arabic} % ab jetzt die normale arabische Nummerierung
\chapter{Einleitung}
Natural Language Processing ist ein großes Feld, welches besonders in der letzten Zeit im Zuge der Digitalisierung viel an Aufmerksamkeit und Wichtigkeit gewonnen hat. Es ermöglicht uns Informationen schneller zu finden, Systeme durch gesprochene Sprache zu steuern oder ganze Texte zu generieren. Eine weitere Aufgabe ist es, eine große Menge an Texten in Kategorien einzuteilen, um die Daten auf eine gewünschte Teilmenge für eine spezifischere Suche oder Analyse zu reduzieren. Die Kategorisierung der Dokumente nach ihrem Inhalt ist hier der häufigste Anwendungsfall, es ist aber auch möglich Texte nach ihrem generellen Genre oder Schreibstil zu vergleichen.

Diese Arbeit wird am Beispiel deutscher Online-Zeitungen untersuchen, welche Möglichkeiten es gibt Texte unabhängig von ihrem Inhalt zu vergleichen. Dabei werden lexikalische, morphologische und syntaktische Merkmale, aber auch die Verwendung inhaltlich irrelevanter Wörter als Features verwendet. Es wird überprüft, inwiefern die Artikel gruppiert werden können und Rückschlüsse auf Unterschiede im Schreibstil ganzer Zeitungen statt nur einzelner Artikel zulassen.

Des Weiteren wird untersucht, ob und wie sich der Schreibstil einer Zeitung je nach Thema, wie z.B. Politik und Sport, unterscheidet.

\chapter{Grundlagen}
Für die Verarbeitung und Analyse von Zeitungsartikeln sind zwei Teilgebiete der Informatik besonders wichtig: Natural Language Processing und Machine Learning. Im Folgenden werden die für die Arbeit relevantesten Konzepte der beiden Bereiche genauer erklärt.

\section{Natural Language Processing}
Natural Language Processing ist ein Teilgebiet der Informatik, das Konzepte und Techniken der künstlichen Intelligenz und des Machine Learning verwendet, um natürliche Sprache zu verarbeiten. Der Begriff natürliche Sprache wird verwendet, um menschliche Sprachen zu beschreiben, die im Gegensatz zu künstlich entwickelten Plansprachen eine historische Entwicklung durchlebt haben. Diese Sprachen befinden sich in einem dauerhaften Entwicklungsprozess und sind häufig sehr variabel in ihrer Verwendung durch den Menschen. Die Analyse der Semantik eines Wortes oder Satzes ist für Computer besonders schwierig, da sich die Bedeutung häufig erst durch den Kontext ergibt.

Mit den schnellen Fortschritten im Bereich des Machine Learning in den letzten Jahrzehnten, eröffneten sich für die Verarbeitung natürlicher Sprache jedoch völlig neue Möglichkeiten. Die Erkennung von Syntax und Semantik wurde damit immer präziser und das Teilgebiet immer relevanter. Gegenwärtig basiert dies hauptsächlich auf Algorithmen des Supervised Learning, für die die Texte vorher manuell mit relevanten Markierungen versehen werden müssen. Ein bekanntes Beispiel für einen Korpus deutscher Sprache mit solchen Annotationen ist der TIGER Corpus \footnote{http://www.ims.uni-stuttgart.de/forschung/ressourcen/korpora/tiger.html}.

\todo{Verwendung nicht annotierter Korpora}

\subsection{Pipeline}
Bei der Analyse eines Textes werden in der Regel verschiedene Schritte abgearbeitet, die jeweils eigene Merkmale der Sprache untersuchen. Es entsteht eine sogenannte Pipeline, die je nach Anwendungsfall unterschiedlich aussieht. Das sequenzielle Ausführen dieser einzelnen Vorgänge ist notwendig, da beispielsweise die Analyse der Syntax voraussetzt, dass das Dokument bereits in Token zerlegt wurde. Im Folgenden werden die für diese Arbeit relevanten Schritte beschrieben.

\todo{eventuell noch andere Modelle beschreiben?}

\subsubsection*{Tokenisierung}
Tokenisierung beschreibt den Prozess, einen gegebenen Text in gleiche Einheiten, sogenannte Token, zu zerlegen. Meistens handelt es sich dabei um Wörter und Satzzeichen, je nach Anwendungsfall können es aber auch Wortgruppen oder Sätze sein. Die Tokenisierung ist häufig eine Voraussetzung einer tiefgehenderen Analyse des Textes, daher ist eine hohe Genauigkeit hier besonders wichtig. Eine Teilung an jedem Satzzeichen um eine Liste von Sätzen zu erhalten, ist zum Beispiel eine triviale Lösung, die bereits gute Ergebnisse erzielt. Es müssen jedoch viele Sonderregeln wie Abkürzungen und Zahlen beachtet werden, sodass das Problem wesentlich komplexer ist ,als es zunächst scheint. Ein Ansatz um höhere Genauigkeit zu erreicheren, ist ein Modell auf Basis bereits mit Annotationen versehener Korpora zu trainieren, welches die Regeln automatisch erlernt.

\subsubsection*{Part-of-speech-Tagging}
Das Part-of-speech-Tagging, auch POS-Tagging, ist ein Verfahren, bei dem jedem Wort oder Satzzeichen die jeweilige Wortart zugeordnet wird. Bei der Analyse ist hierbei vor allem der Kontext in dem das Wort erscheint wichtig, da sich daraus häufig erst die Bedeutung ergibt. Die Informationen über die Wortart und oft auch weitere Details, geben sogenannte Tags, die meist aus einem festen Tagset stammen. In dieser Arbeit werden die Tags aus dem Stuttgart-Tübingen-Tagset (STTS) \footnote{http://www.ims.uni-stuttgart.de/forschung/ressourcen/lexika/TagSets/stts-table.html} verwendet. Der Satz "`Martin findet eine grüne Blechdose."' sieht nach dem POS-Tagging beispielsweise so aus:

Martin/NE findet/VVFIN eine/ART grüne/ADJA Blechdose/NN ./\$.

Die Tags geben Auskunft über mehr als nur die Wortart. Zum Beispiel sind die beiden Wörter \textit{Martin} und \textit{Blechdose} jeweils Nomen. Da jedoch beim POS-Tagging auch die Definition des Wortes überprüft wird, kann \textit{Martin} korrekt als Eigenname mit dem Tag NE identifiziert werden. Weiterhin wurde in diesem Satz die Verbform und der Adjektiv-Typ korrekt erkannt.

\subsubsection*{Dependency Parsing}
Die Analyse der syntaktischen Struktur eines Satzes ist ein weiterer wichtiger Schritt in der Verarbeitung natürlicher Sprache. Beim Dependency Parsing wird zunächst jeder Satz nur auf seine Wörter reduziert, um dann die Beziehungen, sogenannte Dependency Relations, der Wörter innerhalb dieses Satzes zu bestimmen.

\bild{dependency_relations}{\textwidth}{Visualisierung der Beziehungen mit Pfeilen}{Dependency Relations}

Diese Beziehungen ergeben letztendlich einen Baum, der navigiert werden kann und auch Aufschlüsse über die Komplexität eines Satzes zulässt. Die Ergebnisse des Dependency Parsing finden neben der Erkennung semantischer Beziehungen zwischen Wörtern auch noch weitere Anwendungen. So werden sie z.B. von dem Natural Language Processing Tool SpaCy für die Erkennung der Satzenden  verwendet \cite{spacyDependencyParsing}.

\subsubsection*{Lemmatisierung}
Für viele Analysen eines Textes ist es von Vorteil oder sogar notwendig, dass nicht jedes Wort welches unterschiedlich geschrieben wird, als ein anderes Wort behandelt wird. Um dies zu erreichen, werden alle Wörter auf ihre Grundform reduziert. Dieser Prozess heißt Lemmatisierung. Dies ist besonders hilfreich für die Feststellung der Häufigkeit eines Wortes in einem Dokument oder Korpus. So wäre die Frequenz der Wörter \textit{findet}, \textit{fand} und \textit{finden} zunächst jeweils eins. Nach der Lemmatisierung gibt es nur noch das Lemma \textit{finden} mit einer Frequenz von drei. Dies hilft dabei, das Rauschen innerhalb eines Textes zu reduzieren und das tatsächliche Vokabular genauer zu beurteilen.

\section{Machine Learning}
\subsection{Feature Engineering}
Was bedeutet Feature Engineering? Sehr wichtiger Schritt; ausschlaggebend für das letztendliche Resultat. In Natural Language Processing sehr viele verschiedene Ansätze: teilweise Studien die sich nur damit befassen, welche Features es gibt/am besten geeignet sind um xy zu erkennen/erreichen. Anreißen welche Ansätze es gibt und welche in dieser Arbeit nicht berücksichtigt werden. Dieser Prozess wird Feature Selection genannt und ist sehr wichtig aus verschiedenen Gründen: Dimensionalität, Noise Reduction, kürzere Trainingszeiten (aus Wikipedia en).
\subsection{Dimensionsionalitätsreduktion}
Fluch der Dimensionalität: Je mehr Dimensionen es gibt, umso weniger sagt der Raum aus, da die Daten immer weiter voneinander entfernt liegen. Hängt davon ab wieviele Beobachtungen es gibt. Ist fürs Clustering ziemlich wichtig, weil dort die Distanzen wichtig sind. (es gibt auch andere Ansätze wie t-SNE, wo nur Cluster eine Aussagekraft haben und keine Entfernungen). Sehr häufig eingesetztes Verfahren: Hauptkomponentenanalyse (PCA). Features haben dann für sich betrachtet keine Aussagekraft mehr, aber ähnlicher Anteil an Informationen/Varianz bleibt erhalten, obwohl weniger Features. Gut dafür geeignet, Cluster in einem Datensatz zu erkennen. Bei Reduktion auf 2 oder 3 Features kann es dann auch sinnvoll geplottet werden. Dabei gehen aber oft viel zu viele Informationen verloren.
\subsection{Klassifizierung}
Was ist Klassifizierung? Zuordnung einer neuen Beobachtung zu einer Kategorie aus einem vordefinierten Set. Wurde vorher auf Basis eines Trainingssets trainiert, bei denen die Kategorie bekannt ist (supervised learning) (aus Wikipedia en). Wie funktioniert Klassifizierung grob (One vs All, Multinomial...). Welche Klassifizierungsalgorithmen gibt es? Auf jeden Fall keinen besten.

\chapter{Textverarbeitung}
Welche Schritte sind erforderlich, was muss besonders berücksichtigt werden?
\section{Verwendete Tools}
\subsection{Python}
Python ist die am häufigsten für Machine Learning verwendete Sprache und es gibt eine Vielzahl an packages die für nlp/ml optimiert sind. Welche packages wurden für diese Arbeit primär genutzt und warum so gut? Zwei sehr wichtige im Detail, sonst noch Numpy, Pandas und Matplotlib? Anaconda?
\subsection{SpaCy}
Was kann Spacy alles? Warum für Spacy entschieden? (ist für die deutsche Sprache schon sehr ausgereift). Vergleich zu anderen NLP tools und Überblick über die tatsächliche Schnelligkeit auf meiner Maschine beim processen?
\subsection{Scikit-learn}
Was kann sklearn alles? Warum für sklearn entschieden? Wie funktioniert es genau? Vergleiche zu anderen ML tools?
\section{Datenselektion}
Die Auswahl der Daten ist Grundlage für alles, blabla
\subsection{Datensatz}
Woher kommt der Datensatz? Wie groß? Welche Zeitungen? Überblick der Zeitspanne. Illustration wie verschieden die Artikel sind?
\subsection{Normalisierung}
Welche Normalisierungen mussten vor der Aufbereitung durchgeführt werden: Herausfiltern von viel zu kurzen Artikeln/Artikeln ohne Inhalt, Zuordnung von Kategorien zu jedem Artikel, Vereinheitlichung des Formats des Datums (ISO). Oft veröffentlichen Zeitungen auch Artikel einer Newsagentur wie z.B. der dpa, diese müssen herausgefiltert werden, um den Schreibstil einer Zeitung ermitteln zu können (nur Artikel von Autoren der Zeitung sollen berücksichtigt werden). Illustration zum Anteil der dpa Artikel.
\section{Textaufbereitung}
Analyse der Texten mit einem Natural Language Processing Tool, das die zuvor beschriebene Pipeline durchläuft (SpaCy). Vorherige Ansätze?

Nachbereitung der Texte: Entfernen von vermeintlichen Sätzen mit weniger als 4 Wörtern. Musste dort sehr rigoros sein, da jeglicher Inhalt mit enthalten war, so auch Kürzel wie (dpa), der Ort, Autor, Quellen oder Verweise.
\section{Featuregenerierung}
Es gibt eine Vielzahl an Features, die man für einen Text generieren bzw. auswählen kann, ohne dass dabei der Inhalt des Artikels einen Einfluss hat. Verweis zu Studien, die analysiert haben, welche Kombination an Features die besten Resultate liefern.
\subsection{Token-basiert}
Durchschnittliche Satzlänge, durchschnittliche Wortlänge?

Frequenz der n häufigsten Wörter (bei mir 30, 50, 100). Sind dann Wörter wie ['der', 'ich', 'dieser']
\subsection{Lexikalisch}
Type-token ratio: Beurteilt die Reichhaltigkeit des Vokabulars, dafür gibt es noch mehr präzisere Indizes (in der Studie von 2000 sind einige Beispiele). Nachteil: funktioniert besser, je länger die Texte sind und die Artikel sind durchschnittlich nicht besonders lang. Illustration dazu?

Readability: Wieder gibt es hier viele verschiedene Indizes. In dieser Arbeit verwendet wurde der Flesch-Reading Ease Index unter Berücksichtigung dass die Sprache Deutsch ist (gibt noch viel mehr für Englisch). Erforderte die Berechnung, wieviele Silben ein Wort hat (war in der Pipeline nicht mit drin)
\subsection{Morphologisch}
POS unigrams: Die Frequenz der einzelnen Wortarten für jeden Artikel, ist teilweise nur sehr gering und auch hier erhöht sich die Aussagekraft mit der Länge des Artikels.

Lexical Density: Wieviele Wörter des Artikels tragen zum Inhalt bei (Nomen, Verben, Adjektive, Adverbien) im Verhältnis zur Gesamtanzahl an Wörtern? Ähnlich zur Stopword Frequency, die auch berechnet wurde, aber zu sehr mit der lexikalischen Dichte korreliert.
\subsection{Syntaktisch}
Dependency relation unigrams: Die Frequenz der einzelnen Dependency relations für jeden Artikel, ist teilweise nur sehr gering und auch hier erhöht sich die Aussagekraft mit der Länge des Artikels.

\chapter{Datenauswertung}
Wie können die Daten überhaupt ausgewertet werden? Es müssen gleichzeitig die Performance der Features, als auch des Algorithmus zur Klassifizierung analysiert und ausgewertet werden.
\section{Überblick}
Ein paar Plots zeigen, die die Korrelation ausgewählter Features zeigen?
\section{Klassifizierung}
Warum versuche ich die Klassifizierung? Weil wenn Klassifizierung nicht wirklich funktioniert und kein Model trainiert werden kann, welches die Zeitung auf Basis der Daten predicten kann, dann ist auch keine Gruppierung möglich.

Ist zudem eine sehr gute Methode dafür festzustellen, an welchen Features man die Zeitungen am besten voneinander unterscheiden kann und welche eventuell völlig unnötig sind und so bei der Gruppierung nur stören würden.
\subsection{Vorbereitung des Datensatzes}
Für das Training muss der Datensatz in Train/Testset aufgeteilt werden. Welche Aufteilung habe ich hier gewählt und warum?

Die Daten müssen vorher normalisiert oder standardisiert werden.
Kommt allerdings auch auf die ausgewählten Features und den gewählten Algorithmus an.
\subsection{Feature Selection}
Welche Verfahren wurden zur Feature Selection verwendet? Warum? Warum war das überhaupt nötig. Vergleich zu bzw. Verweis auf bisherige(n) Studien zur Feature Selection zur Genre/Author detection.

(evtl. welche Features sind pro Zeitung relevant gewesen? weil unterschiedliche thetas pro Kategorie bei One vs Rest Klassifizierung)
\subsection{Messung der Performance}
Welche verschiedenen Ansätze gibt es hier?
\subsubsection*{F-Maß}
\subsubsection*{Wahrheitsmatrix}
\subsubsection*{Beurteilung der Wahrscheinlichkeiten}
Wahrheitsmatrix kann auch mit den probabilities gemacht werden. Sagt noch mehr über die Sicherheit der Vorhersagen aus.
\subsection{Verwendete Verfahren}
Warum müssen verschiedene Verfahren getestet werden?
\subsubsection*{Linear Discriminant Analysis}
\subsubsection*{Logistische Regression}
\subsubsection*{Random Forest}
\subsubsection*{Support Vector Machines}
\section{Clustering}
Zuerst wurde untersucht, ob beim Clustering der einzelnen Artikeln jeder Zeitung Cluster entstehen, welche die einzelnen Zeitungen repräsentieren. Weiterhin wird überprüft, ob sich dabei Cluster ergeben, die Artikel verschiedener Zeitung haben. Das wäre dann schon ein sehr guter Indikator dafür, dass zwei oder mehr Zeitungen einen ähnlichen Schreibstil haben.

Ein weiterer Ansatz ist es, den Durchschnitt jeder Zeitung zu berechnen und anschließend die Zeitungen zu clustern. Dies hat jedoch Nachteile: Der Durchschnitt einer Zeitung ist nicht besonders repräsentativ, besonders wenn die Standardabweichung hoch ist. Zudem gibt es hier je nach Kategorie nur etwa 8 oder weniger "Beobachtungen" die geclustert werden können.
\subsection{Feature Extraction}
Wie in den Grundlagen bereits beschrieben, ist es beim Clustering besonders wichtig, dass die Dimensionalität nicht hoch ist. Vor allem für die Visualisierung ist es notwendig, die Features auf zwei Dimensionen zu reduzieren.
\subsubsection*{PCA}
Hauptkomponentenanalyse: welche Ergebnisse gibt es hier?
\subsubsection*{t-SNE}
Ein Verfahren, was genau dafür gedacht ist visuell Cluster zu zeigen, bei denen die Distanz untereinander im Plot keine Aussagekraft hat. Wie sehen hier die Ergebnisse aus? Spoiler: KACKEE

\chapter{Ergebnis}

\bibliographystyle{alphadin_martin}
\bibliography{bibliographie}

\chapter*{Erklärung}

Hiermit versichere ich, dass ich die vorliegende Arbeit selbstständig verfasst und keine anderen als die angegebenen Quellen und Hilfsmittel benutzt habe, dass alle Stellen der Arbeit, die wörtlich oder sinngemäß aus anderen Quellen übernommen wurden, als solche kenntlich gemacht und dass die Arbeit in gleicher oder ähnlicher Form noch keiner Prüfungsbehörde vorgelegt wurde.

\vspace{3cm}
Ort, Datum \hspace{5cm} Unterschrift\\

\end{document}